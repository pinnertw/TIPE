\documentclass[12pt, a4paper]{article}
\usepackage[T1]{fontenc}
\usepackage[utf8]{inputenc}
\usepackage{geometry}
\usepackage{graphicx}
\usepackage{wrapfig}
\usepackage{indentfirst}

% Title
\title{Rapport de TIPE\\
Morpion}
\author{Peng-Wei, Chen, MP, \oldstylenums{2017}-\oldstylenums{2018}}

\begin{document}

\maketitle

\section{Introduction}
\subsection*{La règle du morpion}
Deux joueurs jouent sur une grille de taille $15 \times 15$ sur le papier. Chaqu'un prend un symbole et on dessine au tour par tour son symbole sur la grille. Le but est d'aligner 5 symboles verticalement, horizontalement ou en diagonale pour gagner.
Dans le cas généralisé, nous appelons (m, n, k)-jeu où la taille de la grille est $m \times n$ et il faut $k$ symboles dans une ligne pour gagner.
\subsection*{    

\end{document}
