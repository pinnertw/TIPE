\documentclass[12pt, a4paper]{article}
\usepackage[T1]{fontenc}
\usepackage[utf8]{inputenc}
\usepackage[french]{babel}
\usepackage{geometry}
\usepackage{graphicx}
\usepackage{wrapfig}
\usepackage{indentfirst}

% Title
\title{Rapport de TIPE\\
Morpion}
\author{Peng-Wei, Chen, MP, \oldstylenums{2017}-\oldstylenums{2018}}

\begin{document}

\maketitle

\section{Introduction}
\subsection*{La règle du morpion}
Deux joueurs jouent sur une grille de taille $15 \times 15$ sur le papier. Chaqu'un prend un symbole et on dessine au tour par tour son symbole sur la grille. Le but est d'aligner 5 symboles verticalement, horizontalement ou en diagonale pour gagner.
Dans le cas généralisé, nous appelons (m, n, k)-jeu où la taille de la grille est $m \times n$ et il faut $k$ symboles dans une ligne pour gagner.
\subsection*{Méthode}
Il existe deux méthodes pour déterminer s'il existe une telle stratégie:
\begin{itemize}
    \item On cherche tous les cas possibles.
    \item On apparie les points de la grille. Si le deuxième joueur peut toujours prévenir la réussite du premier joueur en jouant le pairage, alors il n'y a pas d'une telle stratégie.
\end{itemize}
Dans le cas où k$\le$7, on utilise la première méthode.
%% introduce determine tree
Or, la complexité de chercher tous les cas possibles est en O($3^{m \times n}$). Ainsi, on utilise une fonction gloutonne pour chercher le cas le \og plus \fg possible. Par exemple, pour la grille ci-dessous, on donne à chaque point une note. 
%% introduce our function
Le point est plus possible si sa note est plus élevée.

%% alpha-beta

%% the second method

%% result


\end{document}
